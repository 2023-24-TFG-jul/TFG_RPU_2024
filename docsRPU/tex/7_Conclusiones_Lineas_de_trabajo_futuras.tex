\capitulo{7}{Conclusiones y Líneas de trabajo futuras}

%Todo proyecto debe incluir las conclusiones que se derivan de su desarrollo. Éstas pueden ser de diferente índole, dependiendo de la tipología del proyecto, pero normalmente van a estar presentes un conjunto de conclusiones relacionadas con los resultados del proyecto y un conjunto de conclusiones técnicas. 
%Además, resulta muy útil realizar un informe crítico indicando cómo se puede mejorar el proyecto, o cómo se puede continuar trabajando en la línea del proyecto realizado. 

	Con todos los apartados anteriores explicados, llega el momento de realizar una valoración global del trabajo realizado. Es difícil valorarse de manera objetiva a uno mismo, pero creo que una actividad muy efectiva para seguir progresando tanto laboralmente como psicológicamente.
	
	Como marcaba al principio de este informe, el trabajo realizado ha sido un verdadero quebradero de cabeza. En muchas ocasiones lo he comparado con la acción de intentar tirar una pared a cabezazos. Con esto último me quiero referir principalmente a lo difícil que ha sido para mí afrontar una aplicación de este nivel.
	
	Es cierto que las herramientas utilizadas permiten realizar infinidad de cosas increíbles, pero creo que la curva de aprendizaje que carga a su espalda apenas tiene pendiente. Aun así, a base de esfuerzo y dedicación, se ha llegado a crear una aplicación operativa que responde a la mayoría de los objetivos definidos en este informe.
	
	El proyecto puede ser escalado de mil formas, pues existen una gran cantidad de nuevas funciones que podría añadir esta aplicación. Si tuviera que marcar alguna en concreto, me quedaría con:
	
	\begin{enumerate}
		\item Inserción de nuevos tipos de subasta: aunque hemos definido la más común y creo firmemente que sería la más utilizada si hubiese más, podríamos tener en cuenta los tres tipos de subastas restantes explicados en la parte teórica de este informe.
		\item Conexión con el entorno Google: muchas aplicaciones y sitios web permiten registrarse con la cuenta de Google, cosa que podría incluirse para aumentar el nivel de profesionalidad de nuestra aplicación.
		\item Diseños más profesionales: la aplicación mostrada es una base de una aplicación que podría incluir diseños y efectos de cambios de vista más espectaculares, dando una experiencia de usuario mucho más satisfactoria.
		\item Perfeccionamiento del modelo: este trabajo puede llevarse la mayor cantidad de tiempo a invertir, pues es fundamental que la predicción de precios sea lo más perfecta posible.
		\item Trabajar en tareas relacionadas con la mejora de la cartera de usuario. Ejemplos de ello podría ser indicadores de transacciones futuras o ingresos directos desde otras aplicaciones.
	\end{enumerate}
	
	En conclusión, las líneas de mejora de esta aplicación son infinitas y tengo claro que se irán alcanzando con el aprendizaje tanto en mi mundo laboral como en el estudiantil, mundos que son un punto y seguido tras la finalización de este trabajo.