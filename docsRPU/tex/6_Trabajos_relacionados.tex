\capitulo{6}{Trabajos relacionados}

	Como tal, no he encontrado un trabajo que sea estrictamente igual a este. Sin embargo, tras hablar y recibir ideas de mis tutores, si que hemos encontrado alguno que podrían compartir características con Kairos.
	
	Uno de los sitios web por excelencia es Ebay, el cual es uno de los mercados en línea más grande. Ebay consiste en un comercio electrónico global donde los usuarios suben sus productos y pueden vendérselos a otros bien a través de subasta o por venta directa. En su caso, no se centra en un único producto, si no que deja al usuario la libertad de vender lo que él decida. Otra aplicación muy parecida es Wallapop, quién sigue la filosofía de Ebay pero no incorpora la posibilidad de vender los productos a través de subasta.
	
	La diferencia de las dos aplicaciones comentadas antes es que no se centran en un producto como los relojes, lo que hace que tengan de todo pero no se especialicen en algo en especial. Por eso nace Kairos, para que las personas ligadas al mundo de los relojes encuentren a Kairos un lugar cómodo para poder comprar y vender lo que más les gusta.
	
	Por otra parte, este trabajo se ha realizado utilizando Flutter y Firestore Database. En cuanto a Flutter, existen muchos Trabajos de Fin de Grado que han utilizado esta herramienta, donde cabe destacar el siguiente relizado por D. Daniel Martín Cubero, a quien referencio y felicito por su trabajo: \cite{daniel:dasell}.
	
	Por último, destacar el trabajo de D. Cesar Linares Traseira, quien creó un modelo de predicción de precios de acciones de bolsa a través de redes nuronales siguiendo un modelo de aprendizaje supervisado. A continuación, referencio a este y felicito su trabajo: \cite{cesar:modelo}.
	
	 
