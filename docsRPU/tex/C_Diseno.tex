\apendice{Especificación de diseño}

\section{Introducción}

\section{Diseño de datos}

\section{Diseño procedimental}

\section{Diseño arquitectónico}

	Siempre es bueno realizar un pequeño esquema de cómo podrían verse las distintas pantallas de nuestra aplicación. Si es cierto que son bocetos que pueden distar bastante de la realidad, pero es una buena práctica si realizamos en conjunto con el cliente. Al final lo que buscamos es entendernos con él lo mejor posible y que mejor que a través de un dibujo.
	
	En mi caso, decidí intentar respresentar las interfaces basándonos en todos los requisitos explicados en apartados anteriores. Para ello, utilicé el sitio web NinjaMock, el cual ya había utilizado en anteriores asignaturas del grado.
	
	NinjaMock es una herramienta muy amigable y eficaz si nuestor objetivo es realizar bocetos de cómo podría quedar el front-end de nuestra aplicación. Aunque tiene una versión de pago, este software nos brinda la oportunidad de crear un único poryecto con todas nuestras interfaces de manera gratuita.
	
	A continucación, adjuntamos las distintas interfaces haciendo breves comentarios de ellas donde fuera necesario:

\imagen{image1}{Diseño inicio de sesión y registro de cuenta}{1}

	Como se puede apreciar, en muchas de las ventanas lanzaremos mensajes a modo de advertencia al usuario si este no completa correctamente los campos requeridos. Todos las condiciones de cada campo se marcan en los requisitos expuestos.
	
\imagen{image2}{Diseño recuperación de contraseña y página principal}{1}

	La idea de la recuperación de contraseña es reactivarla a través de un código al correo electrónico del usuario. La mayoría de los iconos queno llevan a alguna interfaz es porque se han marcado como iconos a ventanas emergentes para ayudar más al usuario.
	
\imagen{image3}{Diseño ver mis relojes y añadir nuevos}{1}

	Los iconos son bastante explícitos. Destaco lo que se marca como un billete: será la puerta a la puesta en venta del reloj. Lo vemos en la siguiente imagen como "creacion de subasta".

\imagen{image4}{Diseño creacion de subasta y estado de ventas}{1}

\imagen{image5}{Diseño de aplicar a una subasta y prediccion de precio}{1}

	Al igual que en otras interfaces, el usuario podrá aplicar a la subasta directamente desde este botón. Saldrá una ventana emergente donde marcará el precio a aplicar.
	
\imagen{image6}{Diseño configuracion de información personal}{1}

	Destacar que no debe cumplimentar todos los campos para configurar la información personal. Aún así, habrá campos que dependan unos de otros como los de contraseña.
