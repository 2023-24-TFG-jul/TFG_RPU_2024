\apendice{Especificación de Requisitos}

\section{Introducción}

	Una buena práctica a la hora de realizar cualquier proyecto es definir previamente que se busca. En cuanto al desarrollo software, es muy importante definir previamente información esencial que, bien estructurada, va a ser de gran ayuda para afrontar mejor el proyecto. Como marcabamos en anteriores apartados, durante la realización del trabajo se ha explicado la metodología ágil a seguir. Sin embargo, existen ciertos conceptos que deben conocerse para entender el porqué de este apartado.	
	
	Una vez reunidos tanto tutores como alumno, se decide definir en grupo cuáles son los requisitos del programa. Denominamos "requisitos funcionales" a todo aquello que esperamos que sea realizado por nuestro programa final. Un ejemplo es: "el programa debe permitir crear una cuenta". Por otro lado, los "requisitos no funcinales" complementan a los anteriores marcando cómo un software debe realizar ciertas actividades. En este caso, un ejemplo sería: "el cambio de una página a otra de la aplicación no puede ser mayor de 4 segundos".
	
	Definir previamente esto nos ayuda a entender mejor que queremos como usuario final y nos permite estructurar mejor el desarrollo del proyecto. Sin embargo, esto es solo un parte. A esto debe sumar la definición de las historias de usuario, los casos de uso y los diagramas de interacción.
	
	Una "historia de usuario" no es más que la explicación del requisito funcional en un lenguaje comprensible para el usuario. Por otra parte, conocemos como "casos de uso" a la forma de describir como el usuario realiza una actividad dentro del sistema, mientras que los "diagramas de interacción" representan als interacciones que se dan dentro de este.
	
	Con todo esto definido, podemos dar comienzo a la definición de objetivos, definición de los requisitos y desarrollo de estos, así como un esquema gráfico de cómo debería ser la aplicación final.

%Una muestra de cómo podría ser una tabla de casos de uso:

% Caso de Uso 1 -> Consultar Experimentos.
%\begin{table}[p]
%	\centering
%	\begin{tabularx}{\linewidth}{ p{0.21\columnwidth} p{0.71\columnwidth} }
%		\toprule
%		\textbf{CU-1}    & \textbf{Ejemplo de caso de uso}\\
%		\toprule
%		\textbf{Versión}              & 1.0    \\
%		\textbf{Autor}                & Alumno \\
%		\textbf{Requisitos asociados} & RF-xx, RF-xx \\
%		\textbf{Descripción}          & La descripción del CU \\
%		\textbf{Precondición}         & Precondiciones (podría haber más de una) \\
%		\textbf{Acciones}             &
%		\begin{enumerate}
%			\def\labelenumi{\arabic{enumi}.}
%			\tightlist
%			\item Pasos del CU
%			\item Pasos del CU (añadir tantos como sean necesarios)
%		\end{enumerate}\\
%		\textbf{Postcondición}        & Postcondiciones (podría haber más de una) \\
%		\textbf{Excepciones}          & Excepciones \\
%		\textbf{Importancia}          & Alta o Media o Baja... \\
%		\bottomrule
%	\end{tabularx}
%	\caption{CU-1 Nombre del caso de uso.}
%\end{table}

\section{Objetivos generales}

	Como indicamos en la presentación formal de la idea para este trabajo, los objetivos de este proyecto podrían resumirse como:
	
\begin{itemize}
	\item Se realizará una aplicación multiplataforma apoyándose en el uso de un SDK denominado Flutter.
	\item La aplicación permitirá al usuario subir su producto y marcar qué tipo de subasta aplicar para la venta de este, así como un precio de venta directa si el comprador no quiere aplicar a la subasta y conseguir el producto directamente
	\item Los tipos de subasta recogidos son: Subasta dinámica ascendente o inglesa, Subasta dinámica descendente o holandesa, Subasta en sobre cerrado de primer precio, y subasta en sobre cerrado de segundo precio o Vickrey
	\item El trabajo incorporará un apartado de recomendación de precios a estipular para la venta directa del objeto basándose en los precios de otros productos ya vendidos, así como subastas ya realizadas.
\end{itemize}

\section{Catálogo de requisitos}

	A modo de ser lo más ordenado posible, se enumeran a continuación todos los requisitos establecidos durante el proyecto:
	
\begin{itemize}
	\item 1) Un usuario accederá a la aplicación introduciendo su email y su contraseña en una pantalla de inicio de sesión.
	\item 2) La pantalla de inicio de sesión mostrará un mensaje de error si el nombre de usuario no existe y/o si la contraseña es errónea.
	\item 3) Un usuario podrá registrarse accediendo desde un botón específico para ello situado en la pantalla de inicio de sesión.	
	\item 4) El registro de un usuario debe contener los siguientes datos personales: nombre, apellidos, fecha de nacimiento, país, email, password y número de cuenta donde se realizarán las transacciones.
	\item 5) Tanto el número de cuenta como el email deben ser únicos en la aplicación.
	\item 6) La contraseña del usuario debe seguir las siguientes directrices: longitud mínima de 8 caracteres y contener al menos una mayúscula, una minúscula, un número y un caracter especial.
	\item 7) Si no se completa de manera correcta el registro, la aplicación lanzará un mensaje de error indicando el origen del problema.
	\item 8) La aplicación contará con una pantalla principal donde dispondrá de las siguientes funcionalidades: mis relojes, estado de mis ventas, compra de un reloj, consultar precio de reloj y configuración.
	\item 9) Un usuario podrá subir su reloj a la aplicación marcando: nombre, modelo, referencia, movimiento, material de la caja, material del brazalete, año de fabricación, estado y género. Son obligatorios marca, modelo, año y estado.
	\item 10) Un usuario podrá eliminar un reloj.
	\item 11) Un usuario podrá editar la información de un reloj.
	\item 12) Un usuario podrá vender su reloj previamente subido a través de uno de los cuatro posibles métodos de subasta explicados anteriormente.
	\item 13) Un usuario podrá vender su reloj de manera directa marcando un precio de venta directa.	
	\item 14) Un usuario podrá eliminar una venta en curso siempre que no se haya recibido una primera respuesta a la subasta.
	\item 15) Un usuario podrá editar una venta en curso siempre que no se haya recibido una primera respuesta a la subasta.
	\item 16) Un usuario podrá comprar un reloj a través de uno de los cuatro posibles métodos de subasta explicados anteriormente.
	\item 17) Un usuario podrá comprar un reloj de manera directa aceptando el precio de venta directa.	
	\item 18) Un usuario podrá editar una compra en curso siempre que otros compradores no hayan respondido a dicha puja.
	\item 19) Un usuario podrá cancelar una puja si esta es la más alta durante la realización de la subasta.
	\item 20) La aplicación integrará una IA que marque cuál es el precio recomendado de venta directa de un producto.
	\item 21) La aplicación contará con un apartado donde se introduciran los siguientes datos para la estimación de un precio de un reloj: nombre, modelo, referencia, movimiento, material de la caja, material del brazalete, año de fabricación, estado y género. Son obligatorios marca, modelo, año y estado.
	\item 22) La aplicación contará con buscadores para llegar a la compras, ventas o productos de manera cómoda.
	\item 23) Se habilitará un modal al hora de fijar un precio de venta una nota donde se marque que la organización se lleva un porcentaje de la venta realizada.
	\item 24) La aplicación contará con un apartado de edición de datos personales.
	\item 25) La aplicación contará con un apartado de edición de contraseña.
\end{itemize}

\section{Especificación de requisitos}


