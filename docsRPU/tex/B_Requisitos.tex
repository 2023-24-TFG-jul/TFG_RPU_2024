\apendice{Especificación de Requisitos}

\section{Introducción}

	Una buena práctica a la hora de realizar cualquier proyecto es definir previamente qué se busca. En cuanto al desarrollo software, es muy importante definir previamente información esencial que, bien estructurada, va a ser de gran ayuda para afrontar mejor el proyecto. Como marcabamos en anteriores apartados, durante la realización del trabajo se ha explicado la metodología ágil a seguir. Sin embargo, existen ciertos conceptos que deben conocerse para entender el porqué de este apartado.	
	
	Una vez reunidos tanto tutores como alumno, se decide definir en grupo cuáles son los requisitos del programa. Denominamos ``requisitos funcionales'' a todo aquello que esperamos que sea realizado por nuestro programa final. Un ejemplo es: ``el programa debe permitir crear una cuenta''. Por otro lado, los ``requisitos no funcinales'' complementan a los anteriores marcando cómo un software debe realizar ciertas actividades. En este caso, un ejemplo sería: ``el cambio de una página a otra de la aplicación no puede ser mayor de 4 segundos''.
	
	Definir previamente esto nos ayuda a entender mejor qué queremos como usuario final y nos permite estructurar mejor el desarrollo del proyecto. Sin embargo, esto es solo un parte. A esto debe sumar la definición de las historias de usuario, los casos de uso y los diagramas de interacción.
	
	Una ``historia de usuario'' no es más que la explicación del requisito funcional en un lenguaje comprensible para el usuario. Por otra parte, conocemos como ``casos de uso'' a la forma de describir como el usuario realiza una actividad dentro del sistema, mientras que los ``diagramas de interacción'' representan als interacciones que se dan dentro de este.
	
	Con todo esto definido, podemos dar comienzo a la definición de objetivos, definición de los requisitos y desarrollo de estos, así como un esquema gráfico de cómo debería ser la aplicación final.

%Una muestra de cómo podría ser una tabla de casos de uso:

% Caso de Uso 1 -> Consultar Experimentos.
%\begin{table}[p]
%	\centering
%	\begin{tabularx}{\linewidth}{ p{0.21\columnwidth} p{0.71\columnwidth} }
%		\toprule
%		\textbf{CU-1}    & \textbf{Ejemplo de caso de uso}\\
%		\toprule
%		\textbf{Versión}              & 1.0    \\
%		\textbf{Autor}                & Alumno \\
%		\textbf{Requisitos asociados} & RF-xx, RF-xx \\
%		\textbf{Descripción}          & La descripción del CU \\
%		\textbf{Precondición}         & Precondiciones (podría haber más de una) \\
%		\textbf{Acciones}             &
%		\begin{enumerate}
%			\def\labelenumi{\arabic{enumi}.}
%			\tightlist
%			\item Pasos del CU
%			\item Pasos del CU (añadir tantos como sean necesarios)
%		\end{enumerate}\\
%		\textbf{Postcondición}        & Postcondiciones (podría haber más de una) \\
%		\textbf{Excepciones}          & Excepciones \\
%		\textbf{Importancia}          & Alta o Media o Baja... \\
%		\bottomrule
%	\end{tabularx}
%	\caption{CU-1 Nombre del caso de uso.}
%\end{table}

\section{Objetivos generales}

	Como indicamos en la presentación formal de la idea para este trabajo, los objetivos de este proyecto podrían resumirse como:
	
\begin{itemize}
	\item Realizar una aplicación multiplataforma apoyándose en el uso de un SDK denominado Flutter.
	\item Permitir al usuario subir su producto y crear una subasta para la venta de este, así como un precio de venta directa si el comprador no quiere aplicar a la subasta y conseguir el producto directamente.
	\item Incorporar un apartado de predicción de precios de un reloj según sus características.
\end{itemize}

	Se detallan con más precisión en el documento memoria de este trabajo.

\section{Catálogo de requisitos}

	A modo de ser lo más ordenado posible, se enumeran a continuación todos los requisitos establecidos durante el proyecto:
	
\subsection{Requisitos funcionales}
	
\begin{itemize}
	\item RF-1) Un usuario accederá a la aplicación introduciendo su email y su contraseña en una pantalla de inicio de sesión.
	\item RF-2) La pantalla de inicio de sesión mostrará un mensaje de error si el nombre de usuario no existe y/o si la contraseña es errónea.
	\item RF-3) Un usuario podrá registrarse accediendo desde un botón específico para ello situado en la pantalla de inicio de sesión.	
	\item RF-4) El registro de un usuario debe contener los siguientes datos personales: nombre, apellidos, fecha de nacimiento, país, email, \emph{password}, número de cuenta donde se realizarán las transacciones y monedero.
	\item RF-5) El email deben ser único en la aplicación.
	\item RF-6) La contraseña del usuario debe seguir las siguientes directrices: longitud mínima de 8 caracteres y contener al menos una mayúscula, una minúscula, un número y un caracter especial.
	\item RF-7) Si no se completa de manera correcta el registro, la aplicación lanzará un mensaje de error indicando el origen del problema.
	\item RF-8) La aplicación contará con una pantalla principal donde dispondrá de las siguientes funcionalidades: listado de mis relojes, listado de subastas y actualización de la información de usuario.
	\item RF-9) Un usuario podrá subir su reloj a la aplicación marcando: nombre, modelo, referencia, movimiento, material de la caja, material del brazalete, año de fabricación, estado y género. Son obligatorios marca, modelo, año y estado.
	\item RF-10) Un usuario podrá eliminar un reloj.
	\item RF-11) Un usuario podrá editar la información de un reloj.
	\item RF-12) Un usuario podrá vender su reloj previamente subido a través de una subasta creada por él.
	\item RF-13) Un usuario podrá vender su reloj de manera directa marcando un precio de venta directa.	
	\item RF-14) Un usuario podrá eliminar una venta en curso siempre que no se haya recibido una primera respuesta a la subasta.
	\item RF-15) Un usuario podrá comprar un reloj pujando una cantidad de dinero a través de una subasta.
	\item RF-16) Un usuario podrá comprar un reloj de manera directa aceptando el precio de venta directa.	
	\item RF-17) La aplicación integrará un modelo que marque cuál es el precio recomendado de venta directa de un producto.
	\item RF-18) La aplicación contará con un apartado donde se introduciran los siguientes datos para la estimación de un precio de un reloj: nombre, modelo, referencia, movimiento, material de la caja, material del brazalete, año de fabricación, estado y género. Son obligatorios marca, modelo, año y estado.
	\item RF-19) La aplicación contará con un apartado de edición de datos personales.
	\item RF-20) La aplicación contará con un apartado de edición de contraseña.
\end{itemize}

\subsection{Requisitos no funcionales}

\begin{itemize}
	\item RNF-1) Los cargos en los monederos deben ser seguros.
	\item RNF-2) La aplicación debe ser intuitiva y al alcance de todo tipo de persona mayor de edad.
\end{itemize}

\section{Especificación de requisitos}

\imagen{casoUsoGeneral}{Esquema general de casos de uso}

\begin{table}[p]
	\centering
	\begin{tabularx}{\linewidth}{ p{0.21\columnwidth} p{0.71\columnwidth} }
		\toprule
		\textbf{CU-1} & \textbf{Iniciar sesión}\\
		\toprule
		\textbf{Versión} & 1.0 \\
		\textbf{Autor} & Rodrigo Pérez Ubierna \\
		\textbf{Requisitos asociados} & RF-1, RF-2 \\
		\textbf{Descripción} & El usuario inicia sesión introduciendo su correo electrónico y su contraseña \\
		\textbf{Precondición} & El usuario debe tener una cuenta registrada. \\
		\textbf{Acciones} &
		\begin{enumerate}
			\def\labelenumi{\arabic{enumi}.}
			\tightlist
			\item El usuario abre la aplicación.
			\item El usuario introduce su email y contraseña donde corresponda.
			\item El usuario pulsa el botón ``Iniciar Sesión''.
		\end{enumerate}\\
		\textbf{Postcondición} & El usuario accede a la pantalla principal de la aplicación. \\
		\textbf{Excepciones} & 
			\begin{itemize}
				\item Si el email no existe, se muestra un mensaje de error.
				\item Si la contraseña es incorrecta, se muestra un mensaje de error.
			\end{itemize} \\
		\textbf{Importancia} & Alta \\
		\bottomrule
	\end{tabularx}
	\caption{Caso de Uso 1: Iniciar sesión}
\end{table}

\begin{table}[p]
	\centering
	\begin{tabularx}{\linewidth}{ p{0.21\columnwidth} p{0.71\columnwidth} }
		\toprule
		\textbf{CU-2} & \textbf{Registrar usuario}\\
		\toprule
		\textbf{Versión} & 1.0 \\
		\textbf{Autor} & Rodrigo Pérez Ubierna \\
		\textbf{Requisitos asociados} & RF-3, RF-4, RF-5, RF-6, RF-7 \\
		\textbf{Descripción} & Un futuro usuario se registra en la aplicación aportando sus datos personales \\
		\textbf{Precondición} & El usuario no estar registrado en la aplicación. \\
		\textbf{Acciones} &
			\begin{enumerate}
				\def\labelenumi{\arabic{enumi}.}
				\tightlist
				\item El usuario accede a la pantalla de inicio de sesión.
				\item El usuario pulsa el botón ``Registrarse''.
				\item El usuario introduce sus datos personales (nombre, apellidos, fecha de nacimiento, país, email, contraseña, número de cuenta, monedero).
				\item El usuario pulsa en el botón ``Registrarse''.
			\end{enumerate}\\
			\textbf{Postcondición} & Se crea la cuenta y ya puede iniciar sesión. \\
			\textbf{Excepciones} &
				\begin{itemize}
					\item Si el email ya está registrado, se muestra un mensaje de error.
					\item Si la contraseña no cumple con los criterios, se muestra un mensaje de error.
					\item Si faltan datos obligatorios, se muestra un mensaje de error.
					\item Si es menor de edad, se muestra un mensaje de error.
				\end{itemize} \\
			\textbf{Importancia} & Alta \\
			\bottomrule
		\end{tabularx}
		\caption{Caso de Uso 2: Registrar usuario}
\end{table}

\begin{table}[p]
	\centering
	\begin{tabularx}{\linewidth}{ p{0.21\columnwidth} p{0.71\columnwidth} }
		\toprule
		\textbf{CU-3} & \textbf{Ver pantalla principal}\\
		\toprule
		\textbf{Versión} & 1.0 \\
		\textbf{Autor} & Rodrigo Pérez Ubierna \\
		\textbf{Requisitos asociados} & RF-8 \\
		\textbf{Descripción} & El usuario puede acceder a la pantalla principal donde verá sus relojes, subastas y opciones de actualización de información. \\
		\textbf{Precondición} & El usuario debe haber iniciado sesión. \\
		\textbf{Acciones} &
		\begin{enumerate}
			\def\labelenumi{\arabic{enumi}.}
			\tightlist
			\item El usuario inicia sesión.
			\item La aplicación muestra la pantalla principal con las funcionalidades disponibles.
		\end{enumerate}\\
		\textbf{Postcondición} & El usuario puede interactuar con las funcionalidades de la pantalla principal. \\
		\textbf{Excepciones} &
			\begin{itemize}
				\item Si el usuario no ha iniciado sesión, se redirige a la pantalla de inicio de sesión.
			\end{itemize} \\
		\textbf{Importancia} & Alta \\
		\bottomrule
	\end{tabularx}
	\caption{Caso de Uso 3: Ver pantalla principal}
\end{table}

\begin{table}[p]
	\centering
	\begin{tabularx}{\linewidth}{ p{0.21\columnwidth} p{0.71\columnwidth} }
		\toprule
		\textbf{CU-4} & \textbf{Registrar reloj}\\
		\toprule
		\textbf{Versión} & 1.0 \\
		\textbf{Autor} & Rodrigo Pérez Ubierna \\
		\textbf{Requisitos asociados} & RF-9 \\
		\textbf{Descripción} & El usuario puede subir un reloj a la aplicación proporcionando los datos necesarios. \\
		\textbf{Precondición} & El usuario debe haber iniciado sesión. \\
		\textbf{Acciones} &
		\begin{enumerate}
			\def\labelenumi{\arabic{enumi}.}
			\tightlist
			\item El usuario accede a la pantalla principal.
			\item El usuario selecciona la opción ver listado de relojes.
			\item El usuario selecciona la opción de subir un reloj.
			\item El usuario introduce los datos del reloj (nombre, marca, modelo, referencia, movimiento, material de la caja, material del brazalete, año de fabricación, estado, género y precio).
			\item El usuario envía el formulario para subir el reloj.
		\end{enumerate}\\
		\textbf{Postcondición} & El reloj es añadido a la colección del usuario en la aplicación.\\
		\textbf{Excepciones} &
			\begin{itemize}
				\item Si faltan datos obligatorios, se muestra un mensaje de error.
			\end{itemize} \\
		\textbf{Importancia} & Alta \\
		\bottomrule
	\end{tabularx}
	\caption{Caso de Uso 4: Registrar reloj}
\end{table}

\begin{table}[p]
	\centering
	\begin{tabularx}{\linewidth}{ p{0.21\columnwidth} p{0.71\columnwidth} }
		\toprule
		\textbf{CU-5} & \textbf{Eliminar reloj}\\
		\toprule
		\textbf{Versión} & 1.0 \\
		\textbf{Autor} & Rodrigo Pérez Ubierna \\
		\textbf{Requisitos asociados} & RF-10 \\
		\textbf{Descripción} & El usuario puede eliminar un reloj de su colección. \\
		\textbf{Precondición} & El usuario debe haber iniciado sesión y tener relojes en su colección. \\
		\textbf{Acciones} &
		\begin{enumerate}
			\def\labelenumi{\arabic{enumi}.}
			\tightlist
			\item El usuario accede a la pantalla principal.
			\item El usuario selecciona la opción ver listado de relojes.
			\item El usuario selecciona el reloj que desea eliminar.
			\item El usuario pulsa el botón ``Eliminar''.
		\end{enumerate}\\
		\textbf{Postcondición} & El reloj es eliminado de la colección del usuario. \\
		\textbf{Excepciones} &
			\begin{itemize}
				\item Si existe algún contratiempo, el reloj no se elimina.
			\end{itemize} \\
		\textbf{Importancia} & Media \\
		\bottomrule
	\end{tabularx}
	\caption{Caso de Uso 5: Eliminar reloj}
\end{table}

\begin{table}[p]
	\centering
	\begin{tabularx}{\linewidth}{ p{0.21\columnwidth} p{0.71\columnwidth} }
		\toprule
		\textbf{CU-6} & \textbf{Editar reloj}\\
		\toprule
		\textbf{Versión} & 1.0 \\
		\textbf{Autor} & Rodrigo Pérez Ubierna \\
		\textbf{Requisitos asociados} & RF-11 \\
		\textbf{Descripción} & El usuario puede editar la información de un reloj en su colección. \\
		\textbf{Precondición} & El usuario debe haber iniciado sesión y tener relojes en su colección. \\
		\textbf{Acciones} &
		\begin{enumerate}
			\def\labelenumi{\arabic{enumi}.}
			\tightlist
			\item El usuario accede a la pantalla principal.
			\item El usuario selecciona la opción ver listado de relojes.
			\item El usuario selecciona el reloj que desea editar.
			\item El usuario pulsa el botón ``Editar''.
			\item El usuario modifica la información del reloj.
			\item El usuario guarda los cambios.
		\end{enumerate}\\
		\textbf{Postcondición} & La información del reloj se actualiza en la colección del usuario. \\
		\textbf{Excepciones} &
			\begin{itemize}
				\item Si faltan datos obligatorios, se muestra un mensaje de error.
			\end{itemize} \\
		\textbf{Importancia} & Media \\
		\bottomrule
	\end{tabularx}
	\caption{Caso de Uso 6: Editar reloj}
\end{table}

\begin{table}[p]
	\centering
	\begin{tabularx}{\linewidth}{ p{0.21\columnwidth} p{0.71\columnwidth} }
		\toprule
		\textbf{CU-7} & \textbf{Vender reloj en subasta}\\
		\toprule
		\textbf{Versión} & 1.0 \\
		\textbf{Autor} & Rodrigo Pérez Ubierna \\
		\textbf{Requisitos asociados} & RF-12\\
		\textbf{Descripción} & El usuario puede vender un reloj creando una subasta. \\
		\textbf{Precondición} & El usuario debe haber iniciado sesión y tener relojes en su colección. \\
		\textbf{Acciones} &
		\begin{enumerate}
			\def\labelenumi{\arabic{enumi}.}
			\tightlist
			\item El usuario accede a la pantalla principal.
			\item El usuario accede a la lista de subastas.
			\item El usuario accede al apartado ``Crear subasta''
			\item El usuario completa los detalles de la subasta (nombre del reloj, precio inicial, precio de venta directa y fecha de cierre).
			\item El usuario confirma la creación de la subasta pulsando en el botón ``Crear''.
		\end{enumerate}\\
		\textbf{Postcondición} & La subasta se crea y el reloj se pone a la venta. \\
		\textbf{Excepciones} &
			\begin{itemize}
				\item Si faltan datos obligatorios para la subasta, se muestra un mensaje de error.
			\end{itemize} \\
		\textbf{Importancia} & Alta \\
		\bottomrule
	\end{tabularx}
	\caption{Caso de Uso 7: Vender reloj en subasta}
\end{table}

\begin{table}[p]
	\centering
	\begin{tabularx}{\linewidth}{ p{0.21\columnwidth} p{0.71\columnwidth} }
		\toprule
		\textbf{CU-9} & \textbf{Eliminar venta en curso}\\
		\toprule
		\textbf{Versión} & 1.0 \\
		\textbf{Autor} & Rodrigo Pérez Ubierna \\
		\textbf{Requisitos asociados} & RF-14 \\
		\textbf{Descripción} & El usuario puede eliminar una venta en curso si no se ha recibido una oferta en la subasta. \\
		\textbf{Precondición} & El usuario debe haber iniciado sesión y tener una venta en curso sin ofertas. \\
		\textbf{Acciones} &
		\begin{enumerate}
			\def\labelenumi{\arabic{enumi}.}
			\tightlist
			\item El usuario accede a la pantalla principal.
			\item El usuario accede a la lista de subastas.
			\item El usuario selecciona la venta que desea eliminar.
			\item El usuario pulsa el botón ``Eliminar Venta''.
		\end{enumerate}\\
		\textbf{Postcondición} & La venta es eliminada y el reloj se retira de la venta. \\
		\textbf{Excepciones} &
			\begin{itemize}
				\item Si la venta ya tiene una oferta, no se puede eliminar y el botón aparece bloqueado.
			\end{itemize} \\
		\textbf{Importancia} & Media \\
		\bottomrule
	\end{tabularx}
	\caption{Caso de Uso 9: Eliminar venta en curso}
\end{table}

\begin{table}[p]
	\centering
	\begin{tabularx}{\linewidth}{ p{0.21\columnwidth} p{0.71\columnwidth} }
		\toprule
		\textbf{CU-10} & \textbf{Pujar en subasta}\\
		\toprule
		\textbf{Versión} & 1.0 \\
		\textbf{Autor} & Rodrigo Pérez Ubierna \\
		\textbf{Requisitos asociados} & RF-15 \\
		\textbf{Descripción} & El usuario puede pujar una cantidad de dinero en una subasta para comprar un reloj. \\
		\textbf{Precondición} & El usuario debe haber iniciado sesión y la subasta debe estar activa. \\
		\textbf{Acciones} &
		\begin{enumerate}
			\def\labelenumi{\arabic{enumi}.}
			\tightlist
			\item El usuario accede a la pantalla principal.
			\item El usuario accede a la lista de subastas.
			\item El usuario selecciona una subasta activa.
			\item El usuario introduce una cantidad para pujar.
			\item El usuario confirma la puja.
		\end{enumerate}\\
		\textbf{Postcondición} & La puja del usuario se registra en la subasta. \\
		\textbf{Excepciones} &
			\begin{itemize}
				\item Si la subasta no está activa, se muestra un mensaje de error.
				\item Si la puja es menor que el mínimo y/o valor actual, se muestra un mensaje de error.
				\item Si la puja es mayor o igual al precio de venta directa, se muestra un mensaje de error.
				\item Si el usuario no tiene suficiente dinero para afrontar todas sus pujas, se muestra un mensaje de error.
				\item Si el usuario tiene menos dinero en su monedero que la cantidad a pujar, se muestra un mensaje de error.
			\end{itemize} \\
		\textbf{Importancia} & Alta \\
		\bottomrule
	\end{tabularx}
	\caption{Caso de Uso 10: Pujar en subasta}
\end{table}


\begin{table}[p]
	\centering
	\begin{tabularx}{\linewidth}{ p{0.21\columnwidth} p{0.71\columnwidth} }
		\toprule
		\textbf{CU-11} & \textbf{Comprar reloj de manera directa}\\
		\toprule
		\textbf{Versión} & 1.0 \\
		\textbf{Autor} & Rodrigo Pérez Ubierna \\
		\textbf{Requisitos asociados} & RF-13, RF-16 \\
		\textbf{Descripción} & El usuario puede comprar un reloj directamente al aceptar el precio de venta directa establecido. \\
		\textbf{Precondición} & El usuario debe haber iniciado sesión y el reloj debe estar a la venta. \\
		\textbf{Acciones} &
		\begin{enumerate}
			\def\labelenumi{\arabic{enumi}.}
			\tightlist
			\item El usuario accede a la pantalla principal.
			\item El usuario accede a la lista de subastas.
			\item El usuario selecciona una subasta.
			\item El usuario confirma la compra directa pulsando sobre el botón de ``Compra directa''.
		\end{enumerate}\\
		\textbf{Postcondición} & El reloj es comprado por el usuario. \\
		\textbf{Excepciones} &
			\begin{itemize}
				\item Si el usuario no tiene suficientes fondos o la transacción no es posible por alguna razón, se muestra un mensaje de error.
				\item Si el usuario no tiene suficiente dinero para afrontar todas sus pujas, se muestra un mensaje de error.
			\end{itemize} \\
		\textbf{Importancia} & Alta \\
		\bottomrule
	\end{tabularx}
	\caption{Caso de Uso 11: Comprar reloj de manera directa}
\end{table}

\begin{table}[p]
	\centering
	\begin{tabularx}{\linewidth}{ p{0.21\columnwidth} p{0.71\columnwidth} }
		\toprule
		\textbf{CU-12} & \textbf{Estimar precio de un reloj}\\
		\toprule
		\textbf{Versión} & 1.0 \\
		\textbf{Autor} & Rodrigo Pérez Ubierna \\
		\textbf{Requisitos asociados} & RF-17 \\
		\textbf{Descripción} & El usuario puede obtener una estimación del precio de un reloj ingresando sus detalles a la hora de querer subirlo a la aplicación. \\
		\textbf{Precondición} & El usuario debe haber iniciado sesión. \\
		\textbf{Acciones} &
		\begin{enumerate}
			\def\labelenumi{\arabic{enumi}.}
			\tightlist
			\item El usuario accede a la pantalla principal.
			\item El usuario selecciona la opción ver listado de relojes.
			\item El usuario selecciona la opción de subir un reloj.
			\item El usuario introduce los datos del reloj (marca, modelo, año de fabricación y estado).
			\item El usuario pulsa sobre el botón ``Predecir precio''.
		\end{enumerate}\\
		\textbf{Postcondición} & El sistema proporciona un mensaje con una estimación del precio del reloj introducido. \\
		\textbf{Excepciones} &
			\begin{itemize}
				\item Si no se pueden calcular los datos necesarios para la estimación, se muestra un mensaje de error.
				\item Si hubiera fallos en la conexión con el modelo y/o la API, se muestra un mensaje de error.
			\end{itemize} \\
		\textbf{Importancia} & Media \\
		\bottomrule
	\end{tabularx}
	\caption{Caso de Uso 12: Estimar precio de un reloj}
\end{table}

\begin{table}[p]
	\centering
	\begin{tabularx}{\linewidth}{ p{0.21\columnwidth} p{0.71\columnwidth} }
		\toprule
		\textbf{CU-13} & \textbf{Editar datos personales}\\
		\toprule
		\textbf{Versión} & 1.0 \\
		\textbf{Autor} & Rodrigo Pérez Ubierna \\
		\textbf{Requisitos asociados} & RF-18 \\
		\textbf{Descripción} & El usuario puede editar sus datos personales en la aplicación. \\
		\textbf{Precondición} & El usuario debe haber iniciado sesión. \\
		\textbf{Acciones} &
		\begin{enumerate}
			\def\labelenumi{\arabic{enumi}.}
			\tightlist
			\item El usuario accede a la pantalla de edición de datos personales.
			\item El usuario modifica los campos deseados.
			\item EL usuario introduce la contraseña a modo de seguridad.
			\item El usuario guarda los cambios.
		\end{enumerate}\\
		\textbf{Postcondición} & Los datos personales del usuario se actualizan en la aplicación. \\
		\textbf{Excepciones} &
			\begin{itemize}
				\item Si hay algún problema con la modificación de datos, se muestra un mensaje de error.
			\end{itemize} \\
		\textbf{Importancia} & Media \\
		\bottomrule
	\end{tabularx}
	\caption{Caso de Uso 13: Editar datos personales}
\end{table}

\begin{table}[p]
	\centering
	\begin{tabularx}{\linewidth}{ p{0.21\columnwidth} p{0.71\columnwidth} }
		\toprule
		\textbf{CU-14} & \textbf{Cambiar contraseña}\\
		\toprule
		\textbf{Versión} & 1.0 \\
		\textbf{Autor} & Rodrigo Pérez Ubierna \\
		\textbf{Requisitos asociados} & RF-19 \\
		\textbf{Descripción} & El usuario puede cambiar su contraseña en la aplicación. \\
		\textbf{Precondición} & El usuario debe haber iniciado sesión. \\
		\textbf{Acciones} &
		\begin{enumerate}
			\def\labelenumi{\arabic{enumi}.}
			\tightlist
			\item El usuario accede a la pantalla de edición de datos personales.
			\item El usuario introduce su contraseña actual y la nueva contraseña.
			\item El usuario confirma la nueva contraseña.
		\end{enumerate}\\
		\textbf{Postcondición} & La contraseña del usuario se actualiza en la aplicación. \\
		\textbf{Excepciones} &
			\begin{itemize}
				\item Si la nueva contraseña no cumple con los requisitos, se muestra un mensaje de error.
			\end{itemize} \\
		\textbf{Importancia} & Alta \\
		\bottomrule
	\end{tabularx}
	\caption{Caso de Uso 14: Cambiar contraseña}
\end{table}


