\capitulo{2}{Objetivos del proyecto}

	Definir los objetivos a la hora de realizar un proyecto no es tarea sencilla, pues un mismo trabajo puede tener enfoques diferentes. Como hablaba en la introducción, esto nace de juntar dos de mis pasiones en un mismo sitio. La oportunidad que me brinda la universidad para combinar ambos mundos es única y, por eso mismo, me gustaría llegar a alcanzar los siguientes objetivos:

\begin{enumerate}
	\item Crear una aplicación intuitiva y amigable para cualquier tipo de usuario.
	\item Estructurar las vistas de la aplicación de forma que el usuario no se pierda en ningún momento y tenga la certeza de que ninguno de sus movimientos es erróneo.
	\item Lanzar una aplicación disponible desde cualquier dispositivo móvil y que responda a los cánones de belleza que una aplicación de este nivel merece.
	\item No entablarse en un solo sistema operativo y conseguir que nuestra aplicación sea multiplataforma.
	\item Conocer las bases de Flutter y aplicar estos conocimientos de la manera más efectiva posible.
	\item Conseguir una comunicación rápida y precisa entre el marco de desarrollo Flutter y la base de datos Firestore Database, consiguiendo así una rapidez comunicativa en el traspaso y operaciones de datos.
	\item 	Tras varios momentos pensando como esta aplicación podría situarse como una clara competidora en el mercado, me di cuenta de que muchas aplicaciones no se centran en lo que de verdad importa: el cliente. Siempre se ha dicho que el cliente tiene la razón, que el cliente debe encontrarse cómodo... pero a la hora de la verdad, las aplicaciones no destinan el esfuerzo suficiente a dejar todo lo más sencillo, amigable y accesible que el cliente merece. Por ello, lo primero que hice fue ponerme en su piel y pensar que me gustaría encontrarme en mi aplicación, que problemas podría a llegar a tener... y di con ello: desconozco cuánto vale mi reloj. Sé que puede sonar raro, pero es así. Son tantos los precios que se estipulan a un mismo reloj que sé, a ciencia cierta, que sería incapaz de marcar un precio de venta directa donde consiguiese el mayor valor de venta posible. De aquí nace la idea de investigar acerca del aprendizaje automático o Machine Learning. Mi objetivo principal es crear un modelo capaz de predecir cuál es el precio más recomendable para la venta de mi reloj, apoyándome en un dataset de más de 280000 relojes de diferentes marcas, estados, géneros, precios...
	\item Crear uan API que permita comunicar nuestra aplicación con el modelo de predicción de precios de manera rápida.
	\item Dar libertad al usuario para poder subir sus relojes y ponerlos en venta, siguiendo unas directrices, pero pudiendo marcar precios y tiempos de finalización según su elección.
	\item Controlar muy bien todos los casos posibles que pueden darse tanto a la hora de subir un reloj como a la hora de crear y llevar a cabo una subasta. El control de esto es fundamental para no experimentar problemas monetarios.
	\item Seguir el patrón de arquitectura Modelo-Vista-Controlador de forma que la aplicación pueda ser escalable de manera sencilla.
\end{enumerate}

	Estos objetivos marcan las bases fundamentales para que la nuestra aplicación sea robusta. A medida que escalemos esta aplicación, el número de objetivos crecerá, consiguiendo cimas que todo programador quiere alcanzar.