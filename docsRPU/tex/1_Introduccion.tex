\capitulo{1}{Introducción}

Kairós no es un trabajo cualquiera. Kairós nace de un hoobie que arrastro desde hace unos años, concretamente desde la herencia de mi primer reloj. Yo no solía fijarme mucho en ellos, pues siempre creí que no era más que una simple herramienta para saber qué hora era en ese momento. Sin embargo, poco a poco decidí informarme sobre este mundo y pase a verlo como lo que era: arte. Un reloj no es un simple accesorio. Un reloj define la identidad de la persona. Muchas personas afirman que un conjunto bonito pierde todo \emph{glamour} si no es acompañado de un reloj en la muñeca. Por esta razón, nunca salgo sin que haya algo que pese en mi brazo.

Actualmente, el coleccionismo de relojes atrae a numerosas personas, especialmente hombres. Podríamos hablar de que civilización se atribuye la invención del reloj, aunque no sería de apoyo al trabajo. Sin embargo, una gran curiosidad es quién portaba los primeros relojes de pulsera en la historia: las mujeres. Los hombres siempre portaban relojes de bolsillo agarrados con una cadena. ¿Curioso verdad? Pues no es hasta la Primera Guerra Mundial cuando los hombres deciden utilizar los relojes de pulsera debido a la comodidad detrás de la trincheras.

¿Y por qué este apartado anterior? La respuesta es sencilla: las principales marcas de relojes hacen colecciones para el género masculino, pues son ellos más propensos a portarlos de manera diaria. De ahí que la gran mayoría de relojes que veremos a lo largo del trabajo lleven la característica del género.

Tras un tiempo informandome, hablando con otros amantes de este mundo... ví que hay un gran obstáculo en esta sociedad cuando hablamos de relojes: la compra y venta de estos productos. No existen apenas sitios web o aplicaciones que se dediquen exclusivamente a la adquisición, bien por subasta o bien por compra directa, de relojes. Por esto, surge Kairós.

Kairós es una aplicación multiplataforma donde cualquier usuario podrá comprar y/o vender su reloj de una manera sencilla. La adquisición o venta de estos se realizará a través de varios tipos de subastas a elegir por el usuario, así como la venta directa del reloj si el usuario lo desease.

Y ahora una de las preguntas que surgen dentro de este entorno: pero, ¿cuánto vale mi reloj? Esta pregunta es la más formulada a la hora de vender esta pieza de arte. Desde Kairós queremos eliminar cualquier duda del usuario y, por ello, la aplicación cuenta con un sistema de predicción del precio del reloj según diversas características propias del accesorio.

Con todo situado, podemos marcar cuáles van a ser los principales apartados de la memoria del trabajo y la dinámica que se va a seguir para su realización:

\begin{enumerate}
	\item Objetivos del proyecto
	\item Conceptos teóricos
	\item Técnicas y herramientas
	\item Aspectos relevantes del desarrollo del proyecto
	\item Trabajos relacionados
	\item Conclusiones y lineas de trabajo futuras
\end{enumerate}

Cada apartado expondrá distintas cosas que se han ido realizando durante el proceso, aunque todos compartiran una misma estructura. En todo ellos se explicarán tanto la parte de la aplicación como la parte del Machine Learning, de forma que todo quede lo más limpio y ordenado posible.

A modo de ser diferente al resto, aprovecho esta parte del informe para desearles que disfruten del trabajo tanto como yo voy a disfrutar contándoselo.
