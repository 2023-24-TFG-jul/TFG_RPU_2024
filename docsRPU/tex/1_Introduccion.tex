\capitulo{1}{Introducción}

¿Sabrían decirme el nombre del dios del tiempo en griego? Si la respuesta es Cronos, están en lo cierto. Si la respuesta es ``que forma más extraña de comenzar una memoria'', también están en lo cierto. Aun así, déjenme centrarme en la primera respuesta. Cronos es el dios del tiempo. Para él, el tiempo es lineal. Y ahora, si les preguntase por el nombre del dios de la oportunidad, ¿qué me responderían? Si saben la respuesta, fantástico; si no, disfruten del título de este trabajo.

Kairos no es un trabajo cualquiera. Kairos nace de un \emph{hooby} que arrastro desde hace unos años, concretamente desde la herencia de mi primer reloj. Yo no solía fijarme mucho en ellos, pues siempre creí que no era más que una simple herramienta para saber qué hora era en un determinado momento. Sin embargo, poco a poco decidí informarme sobre este mundo y pasé a verlo como lo que era: arte. Un reloj no es un simple accesorio. Un reloj define la identidad de la persona. Muchas personas afirman que un conjunto bonito pierde todo \emph{glamour} si no es acompañado de un reloj en la muñeca. Por esta razón, nunca salgo sin que haya algo que pese en mi brazo.

Actualmente, el coleccionismo de relojes atrae a numerosas personas, especialmente hombres. Podríamos hablar de a qué civilización se le atribuye la invención del reloj, aunque no sería de apoyo al trabajo. Sin embargo, una gran curiosidad es quién portaba los primeros relojes de pulsera en la historia: las mujeres. Los hombres siempre portaban relojes de bolsillo agarrados con una cadena. Curioso, ¿verdad? Pues no es hasta la Primera Guerra Mundial cuando los hombres deciden utilizar los relojes de pulsera debido a la comodidad detrás de las trincheras.

¿Y por qué este apartado anterior? La respuesta es sencilla: las principales marcas de relojes hacen colecciones para el género masculino, pues son ellos más propensos a portarlos de manera diaria. De ahí que la gran mayoría de relojes que veremos a lo largo del trabajo lleven la característica del género.

Tras un tiempo informándome, hablando con otros amantes de este mundo, ví que hay un gran obstáculo en esta sociedad cuando hablamos de relojes: la compra y venta de estos productos. No existen apenas sitios web o aplicaciones que se dediquen exclusivamente a la adquisición, bien por subasta o bien por compra directa, de relojes. Por esto, surge Kairos.

Kairos es una aplicación multiplataforma donde cualquier usuario podrá comprar y/o vender su reloj de una manera sencilla. La adquisición o venta de estos se realizará a través de la subasta ascendente, así como por venta directa.

Y ahora una de las preguntas que surgen dentro de este entorno: pero ¿cuánto vale mi reloj? Esta pregunta es la más formulada a la hora de vender esta pieza de arte. Desde Kairos queremos eliminar cualquier duda del usuario y, por ello, la aplicación cuenta con un sistema de predicción del precio del reloj según diversas características propias del accesorio.

Con todo situado, podemos marcar cuáles van a ser los principales apartados de la memoria del trabajo y la dinámica que se va a seguir para su realización:

\begin{enumerate}
	\item Objetivos del proyecto
	\item Conceptos teóricos
	\item Técnicas y herramientas
	\item Aspectos relevantes del desarrollo del proyecto
	\item Trabajos relacionados
	\item Conclusiones y líneas de trabajo futuras
\end{enumerate}

Cada apartado expondrá distintos trabajos que se han ido realizando durante el proceso, aunque todos compartirán una misma estructura. En todo ellos se explicarán tanto la parte de la aplicación como la parte del Machine Learning, de forma que todo quede lo más limpio y ordenado posible. Personalmente es un reto para mí ya que mis conocimientos en cada una de las materias eran ínfimos, pero es lo que tienen los retos: a base de golpes y más golpes se llegará al objetivo final.

Entonces ¿por qué el nombre Kairos? Como les contaba, según la mitología griega, su nombre define al dios de la oportunidad. Kairos era representando con unas alas en sus pies y un par de pelos muy largos. Decían que si pasaba por tu lado, solo debías estar rápido y cogerle de su escasa melena. Esta era mi oportunidad de agarrar a Kairos y empezar a aprender cómo crear y lanzar mi propia aplicación. Disfruten del trabajo.
