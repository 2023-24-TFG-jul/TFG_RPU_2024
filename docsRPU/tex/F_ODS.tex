\apendice{Anexo de sostenibilización curricular}

%\section{Introducción}
%Este anexo incluirá una reflexión personal del alumnado sobre los aspectos de la sostenibilidad que se abordan en el trabajo.
%Se pueden incluir tantas subsecciones como sean necesarias con la intención de explicar las competencias de sostenibilidad adquiridas durante el alumnado y aplicadas al Trabajo de Fin de Grado.
%
%Más información en el documento de la CRUE \url{https://www.crue.org/wp-content/uploads/2020/02/Directrices_Sosteniblidad_Crue2012.pdf}.
%
%Este anexo tendrá una extensión comprendida entre 600 y 800 palabras.

	En la reunión realizada por la Naciones Unidas el 25 de septiembre de 2015 según las Naciones Unidas \cite{ods}, ``los líderes mundiales adoptaron un conjunto de objetivos globales para erradicar la pobreza, proteger el planeta y asegurar la prosperidad para todos como parte de una nueva agenda de desarrollo sostenible.''
	
	La finalidad de cada objetivo es que toda persona habitante del planeta ponga de su parte y trate de realizar cualquier actividad teniendo en cuenta los objetivos estipulados en la reunión. De esta forma, se prevé que en 15 años se consiga un desarrollo de la sostenibilidad que repercutirá en las generaciones venideras.
	
	Desde Kairos, estamos muy concienciados con la causa y creemos firmemente que si todos aportamos nuestro pequeño grano de arena, podremos conseguir de esta sociedad un mundo mejor. Para ser lo más claro posible, se expone como Kairos vela por alcanzar los objetivos exponiendo ejemplos de parte de ellos.
	
\section{Trabajo Decente y Crecimiento Económico}

	El ODS 8, más conocido como Trabajo Decente y Crecimiento Económico, busca ``promover el crecimiento económico inclusivo y sostenible, el empleo y el trabajo decente para todos.''
	
	Kairos sigue la filosofía de dicho objetivo, pues la venta de relojes en la aplicación da lugar a la creación de oportunidades económicas tanto para compradores como para vendedores. Además, si nos centramos en la visión de los vendedores, estos usuarios generan fuentes de ingresos revendiendo los productos adquiridos como nuevos, provocando así el fomento del emprendimiento, y accediendo así a un mercado mucho más amplio y con precios de venta justos (sobre todo gracias a la predicción de precios de los relojes que realiza el modelo de predicción creado).
	

\section{Industria, Innovación e Infraestructura}

	El ODS 9, más conocido como Industria, Innovación e Infraestructura, busca ``construir infraestructuras resilientes, promover la industrialización sostenible y fomentar la innovación.''
	
	La implementación de nuestra aplicación es un claro ejemplo de innovación tecnológica que permite acceder a los mercados (en este caso digital) de una manera más eficiente y sostenible. Además, Kairos puede convertirse en un ejemplo para mercados de segunda mano siendo una inspiración para ellos y consiguiendo su modernización.
	

\section{Reducción de las Desigualdades}

	El ODS 10, más conocido como Reducción de las Desigualdades, estipula que ``la desigualdad amenaza el desarrollo social y económico a largo plazo, frena la reducción de la pobreza y destruye el sentido de realización y autoestima de las personas''.
	
	Kairos se caracteriza por ser una aplicación de venta de relojes de segunda mano donde cualquier persona de cualquier grupo socioeconómico puede participar en dicho mercado. La idea es llevar el mundo de los relojes a todo el planeta, consiguiendo que sectores de la sociedad que no puedan comprar un reloj nuevo, tengan acceso a productos de calidad con precios mucho más reducidos y cambiantes.

	
\section{Promoción y Consumo Responsables}

	El ODS 12, más conocido como Promoción y Consumo Responsables, busca ``garantizar modalidades de consumo y producción sostenibles''. 
	
	La aplicación que se presenta en este proyecto fomenta la reutilización y la extensión del ciclo de vida de los productos, en este caso relojes. Además, debido a la facilidad que supone la creación de subastas, la demanda de productos nuevos disminuye, y por tanto, reduce la producción industrial que supone un gran impacto ambiental.
	
	Con todo esto conseguimos reducir el consumo de recursos naturales, generación de residuos electrónicos y la emisión de gases que puedan fomentar el efecto invernadero.
	

\section{Alianzas para Lograr los Objetivos}

	El ODS 17, más conocido como Alianzas para Lograr los Objetivos, busca ``revitalizar la alianza mundial para el desarrollo sostenible''.
	
	Desde Kairos sabemos que esto es fundamental para crecer como aplicación. La venta de nuestros productos supone un transporte necesario para llevar el reloj del vendedor al comprador. La idea es colaborar con distintas empresas de logísticas que aseguren realizar sus tareas siguiendo unas prácticas ecológicas.

%	Mi paso por la Universidad de Burgos no solo me ha ayudado a encontrarle el gusto a un mundo tan grande como la informática, si no que ha sido una transcurso del tiempo donde he podido madurar como persona. Recuerdo el primer curso de la carrera como si fuera ayer. Entré pensando ser conocedor de mucha parte de la informática por ser bueno con los dispositivos electrónicos, pero en tan solo dos semanas me di cuenta de que no conocía ni una décima parte de lo que la informática abarcaba.
%	
%	Para mí, terminar este trabajo ha sido uno de los logros personales más emocionantes que he experimentado. No solo por saber que es el final, si no por todo lo que ha conllevado. El grado me ha enseñado a mejorar mi trabajo en equipo, cosa que me hizo mejorar en mi vida deportiva. El grado me ha enseñado a aportar soluciones a cualquier problema que me tocara afrontar. Y todo esto se ve reflejado en este trabajo.
%	
%	Si algo define este proyecto es la constancia a no rendirse en ningún momento. A saber que soy yo quien debía decidir cuando acabar y no los problemas personales externos. A valerme por mi mismo y conseguir desplegar una idea propia con la ayuda de mis tutores, a quien doy las gracias y dedico esta línea para decirles una vez más lo fácil y satisfactorio que ha sido trabajar con ellos.
%	
%	Centrándome más en lo aprendido durante el grado y aplicado a este trabajo, creo que no sería totalmente objetivo marcando cuáles han sido las asignaturas que más han influido en este proyecto. Creo firmemente que de todas he sacado tanto conocimientos teóricos como enseñanzas para mi futura vida laboral y personal.
%	
%	En lo que respecta a la parte del \emph{front-end}, debo admitir que la programación \emph{web} ha sido muy escasa a lo largo de estos años. No ha habido ninguna asignatura que se centrase únicamente en ello. Con esto no quiero decir que se deba más importancia a ello, pero si que me parezca extraño que no haya una serie de créditos dedicados a realizar y desplegar una aplicación web, aunque sea mínima. Sin embargo, aquí debo referenciar a la empresa que me dio la oportunidad de realizar mis prácticas curriculares con ellos, pues me enseñaron las bases de todo lo que una aplicación web de tener, sea cual sea.
%	
%	Aun así, el \emph{front-end} no solo es programar. También se deben realizar unas tareas previas que son de vital importancia a la hora de asentar qué es lo que quiere ver el usuario y cómo podemos hacer para que el usuario se sienta cómodo. Un ejemplo es ``Análisis y Diseño de Sistemas''. Sin embargo, si hablamos más de cómo debemos gestionar un proyecto o definir cuáles son los requisitos que el cliente quiere para su aplicación, asignaturas como ``Ingeniería del Software'', ``Gestión de Proyectos'' o ``Validación y Pruebas'' son ejemplos claros. Me gustaría destacar estas dos últimas por la gran aportación que han tenido tanto en mi Trabajo de Fin de Grado como en mi corta vida laboral.
%	
%	En cuanto a la parte más técnica, asignaturas como ``Programación'', ``Metodología de la programación'', ``Estructura de datos'', ``Sistemas inteligentes'' o ``Procesadores del lenguaje'' me han aportado muchos conocimientos. Aunque el lenguaje de programación utilizado en este proyecto es totalmente distinto a lo visto en estas asignaturas, el profesorado ha conseguido que asiente las bases de la programación. Además, como confesión personal, a veces de tanto programar he adquirido la manía de imaginarme lo que veo llevado a la programación y el flujo de datos. Tengo que añadir que, personalmente, junto con las asignaturas ``Sistemas operativos'' y ``Informática Básica'', han sido un verdadero reto y me han servido en mi vida personal para no rendirme nunca y saber que con trabajo se consigue cualquier cosa que uno se proponga, y una de ellas es este proyecto.
%	
%	Por otro lado, si nos centramos en la parte \emph{back-end} del proyecto, las asignaturas ``Bases de datos'' y ``Aplicaciones de Bases de Datos'' han sido las claras protagonistas en esta parte del trabajo. A título personal, encontré en estas asignaturas un mundo que desconocía y que hoy es una de mis ramas preferidas dentro de la informática. Aunque son solo dos asignaturas frente a las casi cuarenta disponibles, la cantidad de conocimientos sobre bases de datos adquirida por mi persona es notable. En cuanto al trabajo, como ya se ha indicado en anteriores apartados, la naturaleza de la base de datos no correspondía con las SQL utilizadas mayoritariamente en estas. Sin embargo, esto no ha sido un impedimento y se ha resuelto todo sin mayor problema.
%	
%	Por último, me gustaría destacar una experiencia que marcó un antes y un después en mi etapa universitaria: el programa ERASMUS. Para mí, fue una etapa de madurez estudiantil impresionante, no solo por el aprendizaje y mejora de distintos idiomas, si no por ver la importancia de estudiar y finalizar una carrera como Ingeniería Informática.